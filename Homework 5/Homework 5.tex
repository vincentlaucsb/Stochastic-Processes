\documentclass[11pt]{article}

\usepackage{graphicx,tikz}
\usepackage{amsmath,amsthm}
\usepackage{amsfonts}
\usepackage{amssymb}
\usepackage{boondox-cal}
\title{ }

\newtheorem*{thm}{Theorem}

\begin{document}
	%\maketitle
	%\date
	\begin{center}	% centers
		\Large{Homework 5}	% Large makes the font larger, put title inside { }
	\end{center}
	\begin{center}
		Vincent La \\
		PSTAT 160A \\
		November 8, 2016
	\end{center}

\begin{enumerate}

\item[10.]
	\begin{enumerate}
		\item[c.] After running a my script a few times I got values of 0.323, 0.3198, 0.32, 0.3129, and 0.3209.
		\item[d.] Using the formula 
		\[\pi_0 = \sum^\infty_{j=0} \pi^j_0 \cdot P_j \]
		We get:
		\[\frac{1}{5} - \frac{4\pi_0}{5} + \frac{\pi^2_0}{2} + \frac{\pi^3_0}{10} = 0\]
		The smallest positive solution to this equation is $\pi_0 = 0.31662$, which is very close to the values I got for c.
		
	\end{enumerate}
\end{enumerate}
\end{document}