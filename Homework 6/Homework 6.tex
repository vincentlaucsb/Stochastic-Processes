\documentclass[11pt]{article}

\usepackage{graphicx,tikz}
\usepackage{amsmath,amsthm}
\usepackage{amsfonts}
\usepackage{amssymb}
\usepackage{boondox-cal}
\title{ }

\newtheorem*{thm}{Theorem}

\begin{document}
	%\maketitle
	%\date
	\begin{center}	% centers
		\Large{Homework 5}	% Large makes the font larger, put title inside { }
	\end{center}
	\begin{center}
		Vincent La \\
		PSTAT 160A \\
		November 8, 2016
	\end{center}

\begin{enumerate}

\item[10.]
	\begin{enumerate}
		\item One run of the script yields:
		\[\pi_1 =  0.2525\]
		\[\pi_2 = 0.2521\]
		\[\pi_3 = 0.2497\]
		\[\pi_4 = 0.2455\]
		\item Solving the system of equations generated by:
		$\pi P = \pi$ where $\pi = (\pi_1,\pi_2,\pi_3,\pi_4)$
		and $\pi_1$ + $\pi_2$ + $\pi_3$ + $\pi_4$ = 1, we get:
		\[\pi_1 = 0.25, \pi_2 = 0.25, \pi_3 = 0.25, \pi_4 = 0.25\]
		\item The proof is left as an exercise for the reader.
		\item Using the method of exact simulation, the state at $X_0$ should be distributed exactly as $\pi$. Here I collected 10000 samples and on one run got these estimated values:
	
		\[\pi_1 =  0.25\]
		\[\pi_2 = 0.2478\]
		\[\pi_3 = 0.2495\]
		\[\pi_4 = 0.2524\]
		
	\end{enumerate}
\end{enumerate}
\end{document}