\documentclass[11pt]{article}

\usepackage{graphicx,tikz}
\usepackage{amsmath,amsthm}
\usepackage{amsfonts}
\usepackage{amssymb}
\usepackage{boondox-cal}
\title{ }

\newtheorem*{thm}{Theorem}

\begin{document}
	%\maketitle
	%\date
	\begin{center}	% centers
		\Large{Homework 4}	% Large makes the font larger, put title inside { }
	\end{center}
	\begin{center}
		Vincent La \\
		PSTAT 160A \\
		October 20, 2016
	\end{center}

\begin{enumerate}

\item[10.]
	\begin{enumerate}
		\item[c.] See output for .py file. Here are some values from one trial:
		\begin{enumerate}
			\item $\mathbb{P}(X_5 = 5)$: 0.037100000000000001
			\item $\mathbb{P}(X_6 = 5)$:  0.1149
			\item $\mathbb{P}(X_7 = 5)$:  0.21560000000000001
			\item $\mathbb{P}(X_8 = 5)$:  0.32579999999999998
			\item $\mathbb{P}(X_9 = 5)$:  0.42970000000000003
			\item $\mathbb{P}(X_10 = 5)$:  0.52649999999999997
			\end{enumerate}
		\item[d.] By exponentiating the transition matrix, I was able to obtain some very close probabilities to my Python estimates:
			\begin{enumerate}
				\item $\mathbb{P}(X_5 = 5)$: 0.0384
				\item $\mathbb{P}(X_6 = 5)$: 0.1152
				\item $\mathbb{P}(X_7 = 5)$: 0.21504
				\item $\mathbb{P}(X_8 = 5)$: 0.32256
				\item $\mathbb{P}(X_9 = 5)$: 0.42706944
				\item $\mathbb{P}(X_10 = 5)$: 0.5225472
			\end{enumerate}
	\end{enumerate}
\end{enumerate}
\end{document}

x